\documentclass[11pt,a4paper]{report}
\usepackage[T1]{fontenc}
\usepackage[spanish]{babel}
\usepackage{amsthm,amsmath,amssymb}
\usepackage{enumerate}
\usepackage{enumitem}


\begin{document}

\begin{titlepage}
    \begin{center}
       \vspace{0.5cm}
        \Large
        UNIVERSIDAD NACIONAL AUTÓNOMA DE MÉXICO
        
        \vspace{0.3cm}
        \Large
        FACULTAD DE ESTUDIOS SUPERIORES ACATLÁN
       
        \vspace{0.3cm}
        \large
        LICENCIATURA EN MATEMÁTICAS APLICADAS Y COMPUTACIÓN
       
        \vspace{1.6cm}
        \normalsize
        Estadística I
        
        \vspace{1.4cm}
        \LARGE 
        \textbf{Tarea Grupal 01}
        
        \vspace{1cm}
        \normalsize
        Presenta
        
        
        \vspace{0.8cm}
        \large
        \textbf{De la Paz Pérez Jocelyne Arizbeth} \\
        423072275 \\
        \vspace{0.5cm}
        \textbf{nombre} \\
        num cuenta
        \vfill
        México,  \today
        
        \vspace{4cm}
        
    \end{center}
\end{titlepage}

\tableofcontents

\chapter{Introducción a la Estadística}



\begin{enumerate}
\item Los siguientes son ejemplos de variables cuantitativas. Indica si son discretas o continuas y si su escala de medición puede ser de intervalo o de razón.


\begin{enumerate}
    \item Talla de calzado de una persona. 
    \item Porcentaje de aprobación de una persona en un cargo público. $\hspace{0.1cm} \rightarrow \hspace{0.1cm} $ continua de razón
    \item Peso en kilogramos de recién nacidos. $\hspace{0.1cm} \rightarrow \hspace{0.1cm} $ continua de razón
    \item Número de individuos restantes de una especie en peligro de ex- tinción. $\hspace{0.1cm} \rightarrow \hspace{0.1cm} $ discreta de razón
    \item La altura de los árboles de cierta especie en un bosque específico. $\hspace{0.1cm} \rightarrow \hspace{0.1cm} $ continua de razón
    \item Consumo de energía eléctrica por casa en una ciudad en un periodo determinado. $\hspace{0.1cm} \rightarrow \hspace{0.1cm} $ continua de razón
    \item Nivel de partículas contaminantes en una ciudad a una hora determinada. $\hspace{0.1cm} \rightarrow \hspace{0.1cm} $ continua de razón
    \item Nivel de coeficiente intelectual de una persona.
    \item Capital de una persona en su cuenta de ahorro para el retiro.
    \item Número de días inhábiles en un calendario escolar.$\hspace{0.1cm} \rightarrow \hspace{0.1cm} $ discrerta de razón
    \item Número de veces que una persona se cepilla los dientes al día. $\hspace{0.1cm} \rightarrow \hspace{0.1cm} $ discreta de razón
    \item Número de mascotas en un hogar. $\hspace{0.1cm} \rightarrow \hspace{0.1cm} $ discreta de razón
    \item El monto solicitado a un banco por una persona para un crédito hipotecario. $\hspace{0.1cm} \rightarrow \hspace{0.1cm} $ continua de razón
    \item El promedio de calificaciones de un alumno. 
\end{enumerate}

\item Supón que las letras SM denotan el salario mínimo que un trabajador gana en un cierto país. Se hace una clasificación de los salarios de todos los trabajadores de acuerdo a las siguientes tres categorías:
    
    \begin{itemize}
        \item \textbf{Salario bajo}: Salario desde un SM y hasta antes de 5 veces el SM.
        \item \textbf{Salario medio}: Salario desde 5 veces el SM y hasta antes de 10 veces el SM.
        \item \textbf{Salario alto}: Salario desde 10 veces el SM en adelante.
    \end{itemize}

    Determina el tipo de variable creada y su escala de medición. Asigna un representante (marca de clase) para cada categoría. \\ \\
    Tipo de variable: cualitativa \\
    Salario bajo: $[1 SM, 5 SM) \hspace{0.1cm} \rightarrow \hspace{0.1cm} \frac{1+5}{2}=3 SM$ \\
    Salario medio: $[5 SM, 10 SM) \hspace{0.1cm} \rightarrow \hspace{0.1cm} \frac{5+10}{2}=7.5 SM$   \\
    Salario alto: $[10 SM, \infty) \hspace{0.1cm} \rightarrow \hspace{0.1cm}$ (no se puede calcular la media)


\item Calcula el cuantil al 25\%, al 50\% y al 75\% de los siguientes conjuntos de
datos.
\begin{enumerate}
    \item 1, 0, 1
    \item  1, 2, 2, 2, 2, 3.
    \item 1, 5, 0, 3, 3, 2, 1, 0.
    \item 2.5, 3.2, 2.7, 2.0, 3.4, 3.7, 5.2, 2.0, 2.5, 4.0, 2.3.
    \item 20, 32, 25, 27, 30, 28, 31, 20, 30.
\end{enumerate}

\item Sea $x_1, \dots, x_n$ un conjunto de datos numéricos con media $\bar{x} \neq 0$. Define $y_i = \frac{x_i}{\bar{x}}$ para $i = 1, 2, \dots, n$. Comprueba que:

    \[
    \text{cv}(y) = \frac{s_x}{|\bar{x}|}
    \begin{cases} 
     \text{cv}(x) & \text{si } \bar{x} > 0, \\
    - \text{cv}(x) & \text{si } \bar{x} < 0.
    \end{cases}
    \]

    donde $\text{cv}(x)$ y $\text{cv}(y)$ representan los coeficientes de variación de $x$ e $y$, respectivamente.

\item Muestra que la curtosis es adimensional e invariante frente a cambios de origen y escala. \\ \\
\begin{enumerate}
    \item Cambios de origen \\ 
    Demostrar que $k(x+a)=k(x)$, siendo $a$ una constante. \\ 
    \[k(x+a)=\frac{1}{s^4}[\frac{1}{n} \sum_{i=1}^n((x_i+a)-(\bar{x}+a))^4]\]
    \[k(x+a)=\frac{1}{s^4}[\frac{1}{n} \sum_{i=1}^n(x_i+a-\bar{x}-a)^4]\]
    \[k(x+a)=\frac{1}{s^4}[\frac{1}{n} \sum_{i=1}^n(x_i-\bar{x})^4]\]
    \[k(x+a)=k(x)\]
    \item  Cambios de escala \\ 
    Demostrar que $k(ax)=k(x)$, siendo $a$ una constante. \\ 
    \[k(ax)=\frac{1}{(as)^4}[\frac{1}{n} \sum_{i=1}^n(ax_i-a\bar{x})^4]\]
    \[k(ax)=\frac{1}{a^4s^4}[\frac{1}{n} \sum_{i=1}^n(a(x_i-\bar{x}))^4]\]
    \[k(ax)=\frac{1}{a^4s^4}[\frac{a^4}{n} \sum_{i=1}^n(x_i-\bar{x})^4]\]
    \[k(ax)=\frac{1}{s^4}[\frac{1}{n} \sum_{i=1}^n(x_i-\bar{x})^4]\]
    \[k(ax)=k(x)\]
\end{enumerate}

\item  Sea $m_k(x)$ el $k$-ésimo momento central de un conjunto de datos $x_1, \dots, x_n$. 
\begin{enumerate}
    \item Sea $c$ una constante y considera la colección de datos trasladados $x_1 + c, \dots, x_n + c$. Comprueba que:
    \[m_k(x + c) = m_k(x).\]
    Comprobación: 
    \[m_k(x + c)=\frac{1}{n} \sum_{i=1}^n((x_i+c)-(\bar{x}+c))^k\]
    \[m_k(x + c)=\frac{1}{n} \sum_{i=1}^n(x_i+c-\bar{x}-c)^k\]
    \[m_k(x + c)=\frac{1}{n} \sum_{i=1}^n(x_i-\bar{x})^k\]
    \[m_k(x + c) = m_k(x)\]

\item Sea $a$ una constante y considera la colección de datos transformados $a x_1, \dots, a x_n$. Comprueba que:
    \[m_k(a x) = a^k \cdot m_k(x).\]
    Comprobación:
    \[m_k(a x) = a^k=\frac{1}{n} \sum_{i=1}^n(ax_i-a\bar{x})^k\]
    \[m_k(a x) = a^k=\frac{1}{n} \sum_{i=1}^n(a(x_i-\bar{x}))^k\]
    \[m_k(a x) = a^k=\frac{1}{n} \sum_{i=1}^n a^k(x_i-\bar{x})^k\]
    \[m_k(a x) = a^k=a^k\cdot (\frac{1}{n} \sum_{i=1}^n (x_i-\bar{x})^k)\]
    \[m_k(a x) = a^k \cdot m_k(x)\]
    
\end{enumerate}
\item Crea la función $asim$ en R que calcule el coeficiente de asimetría de una variable y ejecuta la linea \texttt{asim(c(2.5, 3.2, 2.7, 2.0, 3.4, 3.7, 5.2, 2.0, 2.5, 4.0, 2.3))}. \\
\begin{figure}[h]
            \centering
        \includegraphics[width=0.9\linewidth]{coef_asim.jpeg}
            \label{fig:enter-label}
\end{figure}

\item Crea la función $curt$ en R que calcule la curtosis de una variable y ejecuta la linea \texttt{curt(c(2.5, 3.2, 2.7, 2.0, 3.4, 3.7, 5.2, 2.0, 2.5, 4.0, 2.3))}.
\begin{figure}[h]
            \centering
        \includegraphics[width=0.9\linewidth]{curt.jpeg}
            \label{fig:enter-label}
\end{figure}

\item Considera el siguiente conjunto de datos 
\begin{table}[h]
    \begin{center}
    \begin{tabular}{|l| c| c| r |l| }
        \hline
        Nombre&Edad&Sexo&Estatura&Escolaridad\\
        
        \hline
        \hline
        Pedro&20&M&1.73&Secundaria\\
        Arturo&23&M&1.75&Primaria\\
        Juan&24&M&1.63&Primaria\\
        Iván&24&M&1.95&Secundaria\\
        Roberto&32&M&1.52&Preparatoria\\
        Carmen&32&F&1.60&Preparatoria\\
        Alejandra&32&F&1.65&Maestría\\
        Gabriel&45&F&1.59&Doctorado\\
        Carlos&45&M&1.82&Licenciatura\\
        María&51&F&1.54&Preparatoria\\
        Rocío&57&F&1.77&Preparatoria\\
        \hline
    \end{tabular}
    \end{center}
\end{table}
\begin{enumerate}
    \item Para cada una de las variables identifica si es cualitativa o cuantitativa. Si es cualitativa, establece si su escala de medición es nominal u ordinal. \\ \\
    Nombre: cualitativa nominal \\
    Edad: cuantitativa de razón \\
    Sexo: cualitativa nominal \\
    Estatura: cuantitativa de razón \\
    Escolaridad: cualitativa ordinal \\
    \item Para la variable edad calcula: media, mediana, moda, varianza, $q(0.20)$, $q(0.80)$, rango, rango intercuartil, coeficiente de asimetría y curtosis. \\ \\
    Media:
    \[\bar{x}=\frac{20+23+24+24+32+32+32+45+45+51+57}{11}\]
    \[\bar{x}=\frac{385}{11}=35\]
    Mediana: $32$ \\
    Moda: $32$ \\
    Varianza:
    \[\frac{1}{11}=((20-35)^2+(23-35)^2.........\]
    
\end{enumerate}
\item Considera el siguiente portafolio de inversión compuesto por diferentes clases de activos:

\begin{table}[h]
    \centering
    \begin{tabular}{|l c c l|}
        \hline
         Clase de activo& Ponderación (\%) & Rendimiento (\%) & ETFs \\
         \hline
         Estados Unidos& 20.34 & 12.20 & VTI \\
         Mercados Desarrollados & 9.00 & 4.67 & VEA \\
         Mercados Emergentes & 4.00 & 2.20 & VWO \\
         Bonos de 10 años & 33.33 & 0.97 & IEF \\
         Bienes raíces & 33.33 & 7.06 & VNQ \\
         \hline
    \end{tabular}
\end{table}

\begin{enumerate} 

    \item Calcula la tasa de rendimiento promedio del portafolio. 

    \item Si el valor total del portafolio es \$5.6 millones, ¿cuál es la cantidad invertida en cada clase de activo?
    
    \item ¿Cuál será el valor esperado del portafolio después del segundo año? \end{enumerate}


\end{enumerate}

\chapter{Métodos para la obtención de funciones de variables aleatorias}

\begin{enumerate}
    \item Sea \( Y \) una variable aleatoria con función de densidad de probabilidad dada por
\[
f(y) =
\begin{cases}
    2(1 - y), & \text{si } 0 \leq y \leq 1, \\
    0, & \text{en cualquier otro punto}.
\end{cases}
\]

\begin{enumerate}
    \item Encuentra la función de densidad de \( U_1 = 2Y - 1 \). \newline
    Por el teorema de cambio de variable.
    \[f_Y(y) = f_X(g^{-1}(y)) \left| \frac{d}{dy} g^{-1}(y) \right|\]
    \[2Y-1 = U_{1}\]
    \[Y = \dfrac{U_{1}+1}{2}\]
    Por lo tanto: 
    \[ \rightarrow 2\left(1-\left( \dfrac{U_{1}+1}{2}\right) \right)\left| \dfrac{1}{2}\right| =  \left(1-\left(\dfrac{U_{1}+1}{2}\right)\right) = \left(\dfrac{2-(U_1+1)}{2}\right) = \dfrac{1-U_{1}}{2}\]
    Rango de $f_{U_{1}}$ 
    \[2(0)-1 = -1\]
    \[2(1)-1 = 1\]
    Por lo tanto, la función de densidad de $U_{1}$ queda de la siguiente forma:  \newline
    \[ f(U_{1}) = 
    \begin{cases}
        \dfrac{1-u_{1}}{2}, & \text{si}\hspace{0.2cm} -1 \leq u_{1} \leq 1 \\
        0, & \text{e.o.c}
    \end{cases}
    \]

     \item Encuentra la función de densidad de \( U_2 = 1 - 2Y \).
     Aplicando el teorema de cambio de variable:
     \[1-2Y = U_{2}\]
     \[Y = \dfrac{1-U_{2}}{2}\]
     Por lo tanto, \newline 
     \[ \rightarrow 2\left(1-\left( \dfrac{1-U_{2}}{2}\right) \right)\left| -\dfrac{1}{2}\right| =  \left(\dfrac{2-(1-U_{2})}{2}\right) = \dfrac{U_{2}+1}{2}\]
     Rango de $f_{U_{2}}$ \newline 
     \[1-2(0) = 1\]
     \[1 - 2(1) = -1\]
     Nota que la función de densidad queda como sigue: 
     \[f(U_{2}) = 
     \begin{cases}
     \dfrac{u_{2}+1}{2}, & \text{si} \hspace{0.2cm} -1 \leq u_{2} \leq 1 \\
     0, & \text{e.o.c}
     \end{cases}
     \]
    \item Encuentra la función de densidad de \( U_3 = Y^2 \).
    Por el teorema de cmabio de variable, hacemos lo siguiente: 
    \[Y^{2} = U_{3}\]
    \[Y = \sqrt{U_{3}}\]
    Calculamos la derivada de la inversa con respecto a $U_{3}$ 
    \[\dfrac{dg^{-1}(U_{3})}{dU_{3}} = \dfrac{1}{2\sqrt{U_{3}}} \]
    \[\rightarrow 2(1 - \sqrt{U_3}) \left|\dfrac{1}{2\sqrt{U_{3}}} \right| = \dfrac{1-\sqrt{U_{3}}}{\sqrt{U_{3}}}\]
    Rango de $f_{u_{3}}$ 
    \[(0)^{2} = 0 \hspace{3cm} (1)^{2} = 1\]
    Por lo tanto, nota que la función de densidad queda como sigue: \newline 
    \[f_{U_{3}} = 
    \begin{cases}
        \dfrac{1- \sqrt{u_{3}}}{\sqrt{u_{3}}}, & \text{si}\hspace{0.2cm} 0 \leq u_{3} \leq 1 \\
        0, & \text{e.o.c}
    \end{cases}
    \]
    \item Encuentra \( \mathbb{E}(U_1) \), \( \mathbb{E}(U_2) \) y \( \mathbb{E}(U_3) \) utilizando las funciones de densidad obtenidas para estas variables aleatorias.
   \begin{equation*}
            \mathbb{E}(U_{1}) = \int_{-1}^{1} u_{1} \dfrac{1-u_{1}}{2} \cdot du_{1} = \int_{-1}^{1} u_{1}(1-u_{1})du_{1} = \dfrac{1}{2} \int_{-1}^{1} u_{1}- (u_{1})^{2} du_{1} =
   \end{equation*}
    \[ \dfrac{1}{2} \left[ \dfrac{(u_{1})^{2}}{2} - \dfrac{(u_{1})^{3}}{3}\right] \bigg|_{-1}^{1} = \dfrac{1}{6}\]
    \begin{equation*}
        \mathbb{E}(U_{2}) = \int_{-1}^{1} u_{2} \dfrac{u_{2}+1}{2} du_{2} = \dfrac{1}{2} \int_{-1}^{1} u_{2}(u_{2}+1)du_{2} = \dfrac{1}{2} \int_{-1}^{1} (u_{2})^{2} - u_{2}du_{2} = 
     \end{equation*}
     \[\dfrac{1}{2}\left[ \dfrac{(u_{2})^{3}}{3} + \dfrac{(u_{2})^{2}}{2}\right]\bigg|_{-1}^{1} = \dfrac{(u_{2})^{3}}{6} + \dfrac{(u_{2})^{2}}{4} \bigg|_{-1}^{1}= \dfrac{10}{24}\]
     \begin{equation*}
         \mathbb{E}(U_{3}) = \int_{0}^{1} u_{3} \left( \dfrac{1- \sqrt{u_{3}}}{u_{3}}\right) du_{3}
     \end{equation*}
     simplificamos el integrando:
\[
\frac{1 - \sqrt{u_3}}{u_3} = \frac{1}{u_3} - u_3^{-\frac{1}{2}}
\]

Por lo tanto, la integral se convierte en:

\[
\mathbb{E}(U_3) = \int_0^1 \left( 1 - u_3^{\frac{1}{2}} \right) du_3
\]
\[
\int_0^1 1 \, du_3 = u_3 \Big|_0^1 = 1 - 0 = 1
\]

\[
\int_0^1 u_3^{\frac{1}{2}} \, du_3 = \frac{2}{3} u_3^{\frac{3}{2}} \Big|_0^1 = \frac{2}{3} (1^{\frac{3}{2}} - 0^{\frac{3}{2}}) = \frac{2}{3}
\]
\[
\mathbb{E}(U_3) = 1 - \frac{2}{3} = \frac{1}{3}
\]
\[
\\frac{1}{3} 
\]


\[
f(y) =
\begin{cases}
    \frac{m}{a} y^{m-1} e^{-(y/a)^m}, & \text{si } y > 0, \\
    0, & \text{en cualquier otro punto},
\end{cases}
\]

donde \( a \) y \( m \) son constantes positivas. Esta función de densidad se usa con frecuencia como modelo para la vida útil de sistemas físicos. Suponga que \( Y \) tiene la densidad de Weibull dada. Responde lo siguiente:

\begin{enumerate}
    \item Encuentra la función de densidad de \( U = Y^m \).
    \item Encuentra \( \mathbb{E}(Y^m) \).
\end{enumerate}

\item Encuentra la función de densidad de $X_{(8)}$  para una m.a. de tamaño 10 con distribución $Unif(0,1)$.



\item Encuentra la función de densidad de $X_{(1)}$ para una m.a. de tamaño 5 con distribución $Exp(100)$.



\item Sea $X$ con distribución $Ber(p)$. Demuestra que:
    \begin{enumerate}
        \item $\mathbb{E}(X) = p$, usando $M_X(t)$.
        \item $\mathbb{V}ar(X) = p(1 - p)$, usando $M_X(t)$.
        \item $\mathbb{E}(X^n) = p$, usando $M_X(t)$.
    \end{enumerate}


\item Sea $X$ con distribución $bin(n, p)$. Demuestra que:
    \begin{enumerate}
        \item $\mathbb{E}(X) = np$, usando $M_X(t)$.
        \item $\mathbb{V}ar(X) = np(1 - p)$, usando $M_X(t)$.
    \end{enumerate}


\item Sean $X$ y $Y$ independientes con distribución Poisson con parámetros $\lambda_1$ y $\lambda_1$ respectivamente. Usa la f.g.m. para demostrar que la variable $X + Y$ tiene distribución $Poisson(\lambda_1 + \lambda_2)$.

\item Sean $X$ y $Y$ independientes con distribución $Gamma(n, \lambda)$ y $Gamma(m, \lambda)$
    respectivamente. Usa la f.g.m. para demostrar que la variable $X + Y$
    tiene distribución $Gamma(n + m, \lambda)$.



\item Simula 1000 observaciones de una distribución $T \ de \ Student$ con 4 grados de libertad, a partir de muestras normales 
    \begin{enumerate}
        \item Obtén la media de la muestra.
        \item Obtén la cuasivarianza de la muestra.
        \item Elabora el histograma de la muestra.
    \end{enumerate}

\item Simula 1000 observaciones de una distribución $F_{v_1=4,v_2=2}$ a partir de muestras normales \begin{enumerate}
        \item Obtén la media de la muestra.
        \item Obtén la cuasivarianza de la muestra.
        \item Elabora el histograma de la muestra.
\end{enumerate}

\end{enumerate}


\end{document}

